%%%%%%%%%%%%%%%%%%%%%%%%%%%%%%%%%%%%%%%%%
% "ModernCV" CV and Cover Letter
% LaTeX Template
% Version 1.3 (29/10/16)
%
% This template has been downloaded from:
% http://www.LaTeXTemplates.com
%
% Original author:
% Xavier Danaux (xdanaux@gmail.com) with modifications by:
% Vel (vel@latextemplates.com)
%
% License:
% CC BY-NC-SA 3.0 (http://creativecommons.org/licenses/by-nc-sa/3.0/)
%
% Important note:
% This template requires the moderncv.cls and .sty files to be in the same 
% directory as this .tex file. These files provide the resume style and themes 
% used for structuring the document.
%
%%%%%%%%%%%%%%%%%%%%%%%%%%%%%%%%%%%%%%%%%

%----------------------------------------------------------------------------------------
%	PACKAGES AND OTHER DOCUMENT CONFIGURATIONS
%----------------------------------------------------------------------------------------

\documentclass[12pt,a4paper,qpl]{moderncv} % Font sizes: 10, 11, or 12; paper sizes: a4paper, letterpaper, a5paper, legalpaper, executivepaper or landscape; font families: sans or roman

\moderncvstyle{classic} % CV theme - options include: 'casual' (default), 'classic', 'oldstyle' and 'banking'
\moderncvcolor{purple} % CV color - options include: 'blue' (default), 'orange', 'green', 'red', 'purple', 'grey' and 'black'

\usepackage{lipsum} % Used for inserting dummy 'Lorem ipsum' text into the template

\usepackage[scale=0.75]{geometry} % Reduce document margins
\setlength{\hintscolumnwidth}{3cm} % Uncomment to change the width of the dates column
%\setlength{\makecvtitlenamewidth}{10cm} % For the 'classic' style, uncomment to adjust the width of the space allocated to your name


\usepackage{tgpagella}

\renewcommand{\listitemsymbol}{-~}

%----------------------------------------------------------------------------------------
%	NAME AND CONTACT INFORMATION SECTION
%----------------------------------------------------------------------------------------

\firstname{Max} % Your first name
\familyname{Melching} % Your last name

% All information in this block is optional, comment out any lines you don't need
\title{Curriculum Vitae}
%\address{123 Broadway}{City, State 12345}
%\mobile{(000) 111 1111}
%\phone{(000) 111 1112}
%\fax{(000) 111 1113}
\email{m-melching@web.de}
\homepage{maxmelching.github.io}{maxmelching.github.io} % The first argument is the url for the clickable link, the second argument is the url displayed in the template - this allows special characters to be displayed such as the tilde in this example
%\extrainfo{additional information}
\photo[100pt][0pt]{pictures/passbild2_crop} % The first bracket is the picture height, the second is the thickness of the frame around the picture (0pt for no frame)
%\quote{"A witty and playful quotation" - John Smith}

%----------------------------------------------------------------------------------------

\begin{document}
\makecvtitle % Print the CV title

\section{Personal Information}

\cvitem{\scshape Name}{Max Melching}

\cvitem{\scshape born}{24.05.2000, Hannover (Germany)}

%\cvitem{\scshape Address}{}



\section{Education}

\cventry[0.25\baselineskip]{\scshape since 2022}{M.Sc.~Physics}{Leibniz University Hannover}{Hannover}{}{}

\cvitem[0.5\baselineskip]{}{\textsc{Minor}: Mathematics}


\cventry[0.25\baselineskip]{\scshape 2018 -- 2022}{B.Sc.~Physics}{Leibniz University Hannover}{Hannover}{Final grade: 1,0 (summa cum laude)}{}

\cvitem[0.25\baselineskip]{}{\textsc{Minor}: Computer Science}

\cvitem[0.25\baselineskip]{}{\textsc{Thesis title}: Systematic Differences in the Source Properties of the Third Gravitational-Wave Catalog}

\cvitem[0.5\baselineskip]{}{\textsc{Thesis description}: studying waveform systematics in some of the detected gravitational wave events, focussing on the respective posterior distributions. This included developing criteria to analyse differences and interpret results.}


%\cventry[0.5\baselineskip]{\scshape 2010 -- 2018}{Abitur}{Matthias-Claudius-Gymnasium}{Gehrden}{Final grade: 1,4}{}%Grammer School


%\cvitem[0.5\baselineskip]{\scshape 2006 -- 2010}{Grundschule Am Langen Feld, Gehrden}%Primary School



\section{Work Experience}

\subsection{\scshape Leibniz University Hannover}

%\subsection{Leibniz University Hannover (Institute for Quantum Optics)}

\cvitem[0.25\baselineskip]{\scshape since Apr 2023}{\textbf{Student job}: data analysis and software development the Cold Atom Lab experiment onboard the ISS.}

%\subsection{Leibniz University Hannover (Institute for Solid State Physics)}% \normalsize, \Large

\cvitem[0.5\baselineskip]{\scshape 2021/22}{\textbf{Student job}: tutoring in $"$Mechanics and Heat$"$, a lecture on experimental physics for students in the first semester. This included grading of exercise sheets and the exam, as well as teaching a tutorial.}%{\scshape Oct 2021 -- Feb 2022}



\section{Honours \& Awards}

\cvitem{\scshape 2023/2024}{Deutschlandstipendium by Leibniz University Hannover}

\cvitem{\scshape 2022/2023}{Deutschlandstipendium by Leibniz University Hannover}

\cvitem{\scshape 2020}{Niedersachsenstipendium by Leibniz University Hannover}

\cvitem{\scshape 2018}{Winner of B!g B4ng Challenge by Leibniz University Hannover}

\cvitem{\scshape 2016}{Cambridge First Certificate in English (Level B2)}%{Oxford Test of English}

\cvitem{\scshape 2015}{2.~Landespreis in Bundeswettbewerb Fremdsprachen (Latin)}



\section{Knowledge \& Skills}

\subsection{\scshape Software}

\cvitem[0.25\baselineskip]{\scshape Advanced}{Python, Jupyter, \LaTeX, \texttt{git} (includes GitLab, GitHub)}

\cvitem[0.25\baselineskip]{\scshape Good}{Linux, Mathematica, C}%, Windows}

\cvitem[0.25\baselineskip]{\scshape Intermediate}{MATLAB, Microsoft Office}


\subsection{\scshape Language}

\cvitem{}{German (native Language), English (conversationally fluent)}%, Latin (Latinum)}



\section{Hobbies \& Interests}%{Sonstige Tätigkeiten, Interessen}

\cvlistitem[0.25\baselineskip]{Football and other sports (American Football, Basketball, Darts)}

\cvlistitem[0.25\baselineskip]{Photography (especially landscapes)}

\cvlistitem[0.25\baselineskip]{Space and space travel}

%\cvlistdoubleitem{item1}{item2} \cvitemwithcomment{links}{fett}{kursiv} \emph{}


\end{document}